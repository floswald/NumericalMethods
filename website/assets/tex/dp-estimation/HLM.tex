\documentclass{beamer}% http://ctan.org/pkg/beamer

% insert my preamble


% my default preamble

\usepackage{amsmath}
\usepackage{amssymb}
\usepackage{bm}
\usepackage{mathpazo}
% font setup
\usepackage[sfdefault,lf]{carlito}
\usefonttheme[onlymath]{serif}
\usepackage{xcolor}
\usepackage{tcolorbox}
\usepackage{hyperref}
\usepackage[round]{natbib}
\usepackage{adjustbox}
\usepackage{multicol}
\usepackage{array}
\usepackage{booktabs}
\usepackage{dcolumn}
\renewcommand{\arraystretch}{1.2} 

\newcounter{saveenumi}
\newcommand{\seti}{\setcounter{saveenumi}{\value{enumi}}}
\newcommand{\conti}{\setcounter{enumi}{\value{saveenumi}}}

% \newcommand{\newblock}{}

% \usepackage{enumitem}

% \usepackage{heuristica}
% \usepackage[heuristica,vvarbb,bigdelims]{newtxmath}
% \usepackage[T1]{fontenc}
% \renewcommand*\oldstylenums[1]{\textosf{#1}}

% \usepackage{lmodern}
% \usepackage[T1]{fontenc}

% \usepackage[bitstream-charter]{mathdesign}
% \usepackage[T1]{fontenc}

% \usepackage{cmbright}
% \usepackage[T1]{fontenc}
% \usetheme{Singapore}
% \setbeamertemplate{frametitle}[default][center]

% set some beamer templates
\setbeamertemplate{footline}[frame number] 
% \setbeamertemplate{caption}[numbered]
% \setbeamertemplate{footline}[frame number]{}
% \setbeamertemplate{navigation symbols}{}
% \setbeamertemplate{footline}{}
% different enumerates
\setbeamercolor{item projected}{bg=blue!80!black,fg=white}
\setbeamertemplate{enumerate items}[circle]
\setbeamertemplate{itemize items}[circle]
% define new commands
\newcommand{\red}[1]{{\color{red}{#1}}}
\newcommand{\blue}[1]{{\color{blue}{#1}}}
\newcommand{\bred}[1]{\textbf{\color{red}{#1}}}
\newcommand{\bblue}[1]{\textbf{\color{blue}{#1}}}

\newtcbox{\cyanbox}{on line, arc=1pt,left=1pt,right=1pt,top=1pt,bottom=1pt, colback=cyan!35!white, colframe=cyan!75!black,boxrule=1pt}
\newtcbox{\bluebox}{on line, arc=1pt,left=1pt,right=1pt,top=1pt,bottom=1pt, colback=blue!15!white, colframe=blue!80!black,boxrule=1pt}
\newtcbox{\redbox}{on line, arc=1pt,left=1pt,right=1pt,top=1pt,bottom=1pt, colback=red!5!white, colframe=red!75!black,boxrule=1pt}
\newtcbox{\greybox}{on line, arc=1pt,left=0.2pt,right=0.2pt,top=0.2pt,bottom=0.2pt, colback=black!5!white, colframe=white!75!black,boxrule=0.5pt}
\newcommand{\link}[2]{\greybox{\hyperlink{#1}{\texttt{#2}}}}
\newcommand{\slink}[2]{\greybox{\hyperlink{#1}{{\small\texttt{#2}}}}}

% \renewcommand<>{\item}{\beameroriginal\item\vspace{\stretch{.25}}}

% new variable linewidth environments
\newenvironment{smalli}
{ \begin{itemize}
    \setlength{\itemsep}{1pt}
    \setlength{\parskip}{1pt}
    \setlength{\parsep}{1pt}     }
{ \end{itemize}                  } 
\newenvironment{widei}
{ \begin{itemize}
    \setlength{\itemsep}{10pt}
    \setlength{\parskip}{10pt}
    \setlength{\parsep}{10pt}     }
{ \end{itemize}                  } 
\newenvironment{smalle}
{ \begin{enumerate}
    \setlength{\itemsep}{1pt}
    \setlength{\parskip}{1pt}
    \setlength{\parsep}{1pt}     }
{ \end{enumerate}                  } 
\newenvironment{widee}
{ \begin{enumerate}
    \setlength{\itemsep}{10pt}
    \setlength{\parskip}{10pt}
    \setlength{\parsep}{10pt}     }
{ \end{enumerate}                  } 
\newenvironment{mide}
{ \begin{enumerate}
    \setlength{\itemsep}{5pt}
    \setlength{\parskip}{5pt}
    \setlength{\parsep}{5pt}     }
{ \end{enumerate}                  } 
\newenvironment{midi}
{ \begin{itemize}
    \setlength{\itemsep}{5pt}
    \setlength{\parskip}{5pt}
    \setlength{\parsep}{5pt}     }
{ \end{itemize}                  } 





\begin{document}

\title{%
\begin{tabular}{c}
{\small Graduate Labor Economics}\\
\\ 
Sorting in the Labor Market (in Theory)
 \\
% {\footnotesize Based on \citet{mortensen2005wage} }
{\small Florian Oswald, SciencesPo Paris}
\\
\end{tabular}%
}




% First Page
\frame{\titlepage} 

\frame{\frametitle{Table of contents}\tableofcontents} 

\section{Frictionless Matching}

\subsection{Introduction}

\begin{frame}{Two-Sided Matching}
How does matching differ from standard markets?\bigskip
\begin{widee}
\item There is no price signal (no walrasian auctioneer)
\item Preferences are over agents not over goods.
\item There are indivisibilities. (Cannot match 30\% with person A and 70\% with person B. in general.)
\end{widee}
\end{frame}

\begin{frame}{Two-Sided Matching: Applications}
\begin{midi}
\item Online Dating
\item Market design: doctor assignment to hospitals
\item Kidney Exchange (google Al Roth Kidney Exchange)
\item School Choice: Boston, New York (soon? SciencesPo)
\item \cite{gale1962college}
\begin{itemize}
\item pose problem
\item provide algorithm
\item show existence
\end{itemize}
\end{midi}

\end{frame}

\subsection{Non-Transferrable Utility}
\begin{frame}
\tableofcontents[currentsubsection] 
\end{frame}

\begin{frame}{One-to-One Matching: A Marriage Market}
\begin{midi}
\item Take two disjoint sets $W=\{w_1,\dots,w_p\}$ and $M=\{m_1,\dots,m_n\}$
\item We want to match in pairs $(w_i,m_j)$ and allow for singles.
\item Agents have preferences over members of other sex.
\item This is just an ordered list:
\begin{equation*}
P(m) = w_1,w_3,[m,w_p],\dots,w_2
\end{equation*}
and similar for women.
\end{midi}
\end{frame}

\begin{frame}{One-to-One Matching}
\begin{midi}
\item We denote 
\begin{equation*}
\mathbf{P} = \{P(m_1),\dots,P(m_n), P(w_1),\dots,P(w_p)\}
\end{equation*}
as the preference profile. 
\item The \bred{marriage market} is defined by $(W,M,\mathbf{P})$
\end{midi}
A particular men-to-women allocation is called a \redbox{matching} $\mu$: \bigskip
\begin{tcolorbox}[colback=red!5!white,colframe=red!75!black,title=Definition: Marriage Matching,fonttitle=\bfseries] 
A marriage matching $\mu$ is a one to one correspondence from $W \cup M$ onto itself, i.e. $\mu(\mu(x)) = x$, such that if $\mu(m)\neq m$ then $\mu(m) \in W$ and if $\mu(w)\neq w$ then $\mu(w) \in M$.
\end{tcolorbox}
\end{frame}

\begin{frame}{One-to-One Matching: Blocking $\mu$}
\begin{midi}
\item a matching $\mu$ is \bred{blocked by individual $k$} if $k$ prefers being single to being matched with $\mu(k)$
\item We write $k \succ_k \mu(k)$.
\item A matching $\mu$ is \bblue{individually rational} if each agent in $\mu$ is \emph{acceptable}, i.e. $\mu$ is not \bred{blocked} by any agent.
\item A matching $\mu$ is blocked by a \bred{pair of agents} $(m,w)$ if 
\begin{equation*}
w\succ_m \mu(m) \text{ and } m\succ_w \mu(w)
\end{equation*}
\end{midi}
\end{frame}

\begin{frame}{One-to-One Matching: Stable Matching}
\begin{tcolorbox}[colback=red!5!white,colframe=red!75!black,title=Definition: Stable Matching,fonttitle=\bfseries] 
A marriage matching $\mu$ is \textbf{stable} if it is not blocked by any individual or any pair of agents.
\end{tcolorbox}
\bigskip
\bigskip
\begin{tcolorbox}[colback=blue!5!white,colframe=blue!75!black,title=Theorem: Gale and Shapley (1962),fonttitle=\bfseries]A stable matching exists for every marriage market. 
\end{tcolorbox}
\end{frame}


\begin{frame}{One-to-One Matching: Proof}
\begin{midi}
\item Their proof uses the \cyanbox{Deferred Acceptance Algorithm} (DAA).
\item Start with one side of the market (men, say):
	\begin{enumerate}
		\item[Iter 1] \hspace{1cm}
		\begin{enumerate}
			\item[i.] Each man proposes to his first choice (if any acceptable ones)
			\item[ii.] Each women holds their most preferred proposer
		\end{enumerate}
		\item[Iter K] ...
		\item[Iter K+L] STOP if no further proposals are made and match any woman to the man whose proposal she is currently holding.
	\end{enumerate}
\item Break ties arbitrarily
\item With finite set of men and women, this algo is finite and always stops.
\end{midi}
\end{frame}

\begin{frame}{One-to-One Matching: Proof}
\begin{midi}
\item Gives rise to a stable matching.
\item Suppose not. Suppose $m$ can do better, i.e. $m$ prefers $w$ to current match $\mu(m)$:
\begin{enumerate}
\item $w\succ_m \mu(m)$
\item $m$ must have proposed to $w$ before proposing to $\mu(m)$
\item $m$ must have been \bred{rejected} by $w$
\item that means that $\mu(w) \succ_w m$
\item \bred{Not} a blocking pair.
\item Match is stable.
\end{enumerate}
\end{midi}
\end{frame}


\begin{frame}{DAA Example}
\begin{midi}
\item Example: Consider market $(W,M,\mathbf{P})$ where
\begin{equation*}
\begin{array}{cc}
P(m_1) = w_1,w_2,w_3,w_4 & P(w_1) = m_2,m_3,m_1,m_4,m_5 \\ 
P(m_2) = w_4,w_2,w_3,w_1 & P(w_2) = m_3,m_1,m_2,m_4,m_5 \\ 
P(m_3) = w_4,w_3,w_1,w_2 & P(w_3) = m_5,m_4,m_1,m_2,m_3 \\ 
P(m_4) = w_1,w_4,w_3,w_2 & P(w_4) = m_1,m_4,m_5,m_2,m_3 \\ 
P(m_5) = w_1,w_2,w_4,m_5 & 

\end{array}
\end{equation*}
\item The DAA proceeds as follows:
\begin{table}
\renewcommand*{\arraystretch}{1}
\begin{tabular}{D{.}{.}{1.5}@{} D{.}{.}{8.3}@{} D{.}{.}{4.4}@{} D{.}{.}{4.4}@{}  D{.}{.}{4.4}@{}  D{.}{.}{4.4}@{} }
  \toprule 
   \multicolumn{1}{l}{Iterate}  &  w_1         & w_2   & w_3 & w_4      & (m_i) \\
  \midrule 
   1.                           &  m_1,m_4,m_5 &       &      & m_2,m_3 &  \\
  \bottomrule
\end{tabular}
\end{table}
\end{midi}
\end{frame}

\begin{frame}{DAA Example}
\begin{midi}
\item Example: Consider market $(W,M,\mathbf{P})$ where
\begin{equation*}
\begin{array}{cc}
P(m_1) = w_1,w_2,w_3,w_4 & P(w_1) = m_2,m_3,m_1,m_4,m_5 \\ 
P(m_2) = w_4,w_2,w_3,w_1 & P(w_2) = m_3,m_1,m_2,m_4,m_5 \\ 
P(m_3) = w_4,w_3,w_1,w_2 & P(w_3) = m_5,m_4,m_1,m_2,m_3 \\ 
P(m_4) = w_1,w_4,w_3,w_2 & P(w_4) = m_1,m_4,m_5,m_2,m_3 \\ 
P(m_5) = w_1,w_2,w_4,m_5 & 
\end{array}
\end{equation*}
\item The DAA proceeds as follows:
\begin{table}
\renewcommand*{\arraystretch}{1}
\begin{tabular}{D{.}{.}{1.5}@{} D{.}{.}{8.3}@{} D{.}{.}{4.4}@{} D{.}{.}{4.4}@{}  D{.}{.}{4.4}@{}  D{.}{.}{4.4}@{} }
  \toprule 
   \multicolumn{1}{l}{Iterate}  &  w_1         & w_2   & w_3 & w_4      & (m_i) \\
  \midrule 
   1.                           &  m_1,m_4,m_5 &       &      & m_2,m_3 &  \\
   2.                           &  m_1         &  m_5     &  m_3    & m_4,m_2 &  \\
  \bottomrule
\end{tabular}
\end{table}
\end{midi}
\end{frame}

\begin{frame}{DAA Example}
\begin{midi}
\item Example: Consider market $(W,M,\mathbf{P})$ where
\begin{equation*}
\begin{array}{cc}
P(m_1) = w_1,w_2,w_3,w_4 & P(w_1) = m_2,m_3,m_1,m_4,m_5 \\ 
P(m_2) = w_4,w_2,w_3,w_1 & P(w_2) = m_3,m_1,m_2,m_4,m_5 \\ 
P(m_3) = w_4,w_3,w_1,w_2 & P(w_3) = m_5,m_4,m_1,m_2,m_3 \\ 
P(m_4) = w_1,w_4,w_3,w_2 & P(w_4) = m_1,m_4,m_5,m_2,m_3 \\ 
P(m_5) = w_1,w_2,w_4,m_5 & 
\end{array}
\end{equation*}
\item The DAA proceeds as follows:
\begin{table}
\renewcommand*{\arraystretch}{1}
\begin{tabular}{D{.}{.}{1.5}@{} D{.}{.}{8.3}@{} D{.}{.}{4.4}@{} D{.}{.}{4.4}@{}  D{.}{.}{4.4}@{}  D{.}{.}{4.4}@{} }
  \toprule 
   \multicolumn{1}{l}{Iterate}  &  w_1         & w_2   & w_3 & w_4      & (m_i) \\
  \midrule 
   1.                           &  m_1,m_4,m_5 &       &      & m_2,m_3 &  \\
   2.                           &  m_1         &  m_5     &  m_3    & m_4,m_2 &  \\
   3.                           &  m_1         &  m_2,m_5     &  m_3    & m_4 &  \\
  \bottomrule
\end{tabular}
\end{table}
\end{midi}
\end{frame}

\begin{frame}{DAA Example}
\begin{midi}
\item Example: Consider market $(W,M,\mathbf{P})$ where
\begin{equation*}
\begin{array}{cc}
P(m_1) = w_1,w_2,w_3,w_4 & P(w_1) = m_2,m_3,m_1,m_4,m_5 \\ 
P(m_2) = w_4,w_2,w_3,w_1 & P(w_2) = m_3,m_1,m_2,m_4,m_5 \\ 
P(m_3) = w_4,w_3,w_1,w_2 & P(w_3) = m_5,m_4,m_1,m_2,m_3 \\ 
P(m_4) = w_1,w_4,w_3,w_2 & P(w_4) = m_1,m_4,m_5,m_2,m_3 \\ 
P(m_5) = w_1,w_2,w_4,m_5 & 
\end{array}
\end{equation*}
\item The DAA proceeds as follows:
\begin{table}
\renewcommand*{\arraystretch}{1}
\begin{tabular}{D{.}{.}{1.5}@{} D{.}{.}{8.3}@{} D{.}{.}{4.4}@{} D{.}{.}{4.4}@{}  D{.}{.}{4.4}@{}  D{.}{.}{4.4}@{} }
  \toprule 
   \multicolumn{1}{l}{Iterate}  &  w_1         & w_2   & w_3 & w_4      & (m_i) \\
  \midrule 
   1.                           &  m_1,m_4,m_5 &       &      & m_2,m_3 &  \\
   2.                           &  m_1         &  m_5     &  m_3    & m_4,m_2 &  \\
   3.                           &  m_1         &  m_2,m_5     &  m_3    & m_4 &  \\
   4.                           &  m_1         &  m_2     &  m_3    & m_4 & m_5 \\
  \bottomrule
\end{tabular}
\end{table}
\end{midi}
\end{frame}

\begin{frame}{DAA Example - $M$ stable matching}
\begin{midi}
\item Example: Consider market $(W,M,\mathbf{P})$ where
\begin{equation*}
\begin{array}{cc}
P(m_1) = w_1,w_2,w_3,w_4 & P(w_1) = m_2,m_3,m_1,m_4,m_5 \\ 
P(m_2) = w_4,w_2,w_3,w_1 & P(w_2) = m_3,m_1,m_2,m_4,m_5 \\ 
P(m_3) = w_4,w_3,w_1,w_2 & P(w_3) = m_5,m_4,m_1,m_2,m_3 \\ 
P(m_4) = w_1,w_4,w_3,w_2 & P(w_4) = m_1,m_4,m_5,m_2,m_3 \\ 
P(m_5) = w_1,w_2,w_4,m_5 & 
\end{array}
\end{equation*}
\item Hence, the $M$-stable matching is:
\begin{equation*}
\mu_M = \begin{array}{ccccc}
w_1 & w_2 & w_3 & w_4 & (m_5) \\
m_1 & m_2 & m_3 & m_4 & (m_5) 
\end{array}
\end{equation*}
\end{midi}
\end{frame}

\begin{frame}{DAA Example - $W$ stable matching}
\begin{midi}
\item Notice that if women were to make proposals, we'd get
\item Hence, the stable matching is:
\begin{equation*}
\mu_W = \begin{array}{ccccc}
w_1 & w_2 & w_3 & w_4 & (m_5) \\
m_2 & m_3 & m_4 & m_1 & (m_5) 
\end{array}
\end{equation*}
\item Implications:
\begin{enumerate}
\item In general, the set of stable matchings is not a singleton.
\item All $m$ weakly prefer $\mu_M$, opposite for women.
\item I.e. there is a conflict between both sides of the market as to who is to make the offer!
\end{enumerate}
\end{midi}
\end{frame}

\begin{frame}{One-to-one Matching Gale and Shapley}
\begin{tcolorbox}[colback=red!5!white,colframe=red!75!black,title=Theorem (Gale and Shapley),fonttitle=\bfseries] 
When all men and women have strict preferences, there always exists an $M$-optimal stable matching, and a $W$-optimal stable matching. Furthermore, the matching $\mu_M$ produced by the DAA with men proposing is the $M$-optimal stable matching. The $W$-optimal stable matching is the matching $\mu_W$ produced by the DAA when women propose.
\end{tcolorbox}

\end{frame}

\begin{frame}{DDA in practice}

look at the example!

\end{frame}

\subsection{Transferrable Utility and Assortative Matching}
\begin{frame}
\tableofcontents[currentsubsection] 
\end{frame}

\begin{frame}{Two-sided Matching with Transferrable Utility}
\begin{midi}
\item Less attractive agents may compensate more attractive ones to form a match
\item in the labor market: Wage.
\item cleaning for roommates, child care in marriage
\item We will now focus on \bred{assortative matching}
\end{midi}
\end{frame}

\begin{frame}{Assortative Matching}
\textbf{Environment:}
\begin{midi}
\item A fixed measure of workers indexed by $x \in \mathbb{X}$ (uniform)
\item A fixed measure of jobs indexed by $y \in \mathbb{Y}$ (uniform)
\item A production function $f(x,y)$
\item Common ranking $f_x>0, f_y>0$
\item The \bred{cross partial derivatives} of $f$ have a key function for monotone matching.
\begin{itemize}
\item Example 1: $f^+(x,y) = \alpha x^\theta y^\theta$
\item Example 2: $f^-(x,y) = \alpha x^\theta (1-y)^\theta + g(y)$
\end{itemize}
\item We allow matched agents to transfer each other $w$ (the wage).
\end{midi}
\end{frame}


\begin{frame}{Assortative Matching}

\textbf{Preferences:}
\begin{midi}
\item Workers care about the wage
\item Firm care about profits: $\pi(y) = f(x,y) - w$
\end{midi}
\medskip
\textbf{Allocation is defined by a matching rule $(\mu,w)$:}
\begin{midi}
\item $\mu(x) = y$: Which worker matches to which firm. Pure matching.
\item $w(x)$: a wage schedule.
\end{midi}
\end{frame}


\begin{frame}{Assortative Matching: equlibrium}
\textbf{Stable Matching Rule:}
\begin{itemize}
\item No pair $(x,y)$ can do better than in equlibrium:
\begin{equation*}
\forall x,y: \quad \underbrace{w(x)}_{x\text{ eqm payoff}} + \underbrace{\pi(\mu^{-1}(y),y)}_{y \text{ eqm payoff}} \geq \underbrace{f(x,y)}_\text{potential output}
\end{equation*}
\end{itemize}

\textbf{Results:}
\begin{midi}
\item Existence: Yes. Shapley and Shubik 1971
\item Eficiency: Yes. Maximizes joint utility
\item Unique: Matching is generically unique, transfers are not
\item Stable Matching and Competitive Eqm coincide (Gretsky, Ostroy and Zame 1999)
\end{midi}
\end{frame}


\begin{frame}{Competitive Eqm and Assortative Matching}
\begin{midi}
\item Firm's problem:
\begin{midi}
\item Take the wage schedule given and choose $x$ to max profit:
\begin{equation*}
\max_x f(x,y) - w(x)
\end{equation*}
\item FOC: $f_x(x,y) - w_x(x) = 0$
\item What is eqm allocation?
\item follows from SOC: $f_{xx}(x,y) - \underbrace{w_{xx}(x)}_\text{?} < 0$
\end{midi}
\end{midi}
\end{frame}

\begin{frame}{Competitive Eqm and Assortative Matching}
\begin{midi}
\item What's the sign of $w_{xx}(x)$? Take derive of FOC at the Eqm condition $\mu(x) = y$:
\begin{align*}
\frac{d}{dx} \left( f_x(x,\mu(x)) - w_x(x) \right) &= 0 \\
                f_{xx}(x,\mu(x)) + f_{xy}(x,\mu(x))\frac{d\mu(x)}{dx} - w_{xx}(x) &= 0 \\
\end{align*}
\item so, the SOC is satisfied provided:
\begin{align*}
f_{xx}(x,y) - w_{xx}(x) < 0 &\iff\\
f_{xy}(x,\mu(x))\frac{d\mu(x)}{dx} > 0 &
\end{align*}
\item Notice that $f_{xy}(x,\mu(x))\frac{d\mu(x)}{dx}$ measures the \bred{assortative matching relationship}
\end{midi}
\end{frame}

\begin{frame}{Production Function and Assortative Matching}
\begin{midi}
\item We have: 
\begin{mide}
\item \redbox{$+$ Assortative Matching (PAM):} $f_{xy}(x,\mu(x)) >0 \text{ if }\frac{d\mu(x)}{dx}>0$
\item \bluebox{$-$ Assortative Matching (NAM):} $f_{xy}(x,\mu(x)) <0 \text{ if }\frac{d\mu(x)}{dx}<0$
\end{mide}
\pause
\item $f_{xy}$ describes the \cyanbox{supermodularity} of $f$.
\item A function $f: \mathbb{R}^k \to \mathbb{R}$ is \bblue{supermodular} if
\begin{equation*}
f(x \uparrow y) + f(x \downarrow y) \geq f(x ) + f(y)
\end{equation*}
where $\uparrow,\downarrow$ denote element-wise $\max,\min$ respectively.
\item If $f$ is twice differentiable, the condition is equivalent to
\begin{equation*}
\frac{\partial^2 f}{\partial z_i\partial z_j} \geq 0,\forall i\neq j.
\end{equation*}
\end{midi}
\end{frame}


\begin{frame}{Production Function and Assortative Matching}
\begin{midi}
\item We have: 
\begin{mide}
\item \redbox{$+$ Assortative Matching (PAM):} $f_{xy}(x,\mu(x)) >0 \text{ if }\frac{d\mu(x)}{dx}>0$
\item \bluebox{$-$ Assortative Matching (NAM):} $f_{xy}(x,\mu(x)) <0 \text{ if }\frac{d\mu(x)}{dx}<0$
\end{mide}
\item $f_{xy}$ describes the \cyanbox{supermodularity} of $f$.
\begin{itemize}
\item if $f$ is super-modular, better workers in better firms is more efficient
\item Gives a clear rationale for why better workers should assortatively match with firms.
\end{itemize}
\item Supermodularity is about the rate of change in the change: Do better workers gain \emph{more} from moving to better firms.
\item Note: With pure matching (like here), we cannot differentiate worker from firm effects.
\end{midi}
\end{frame}


\section{Matching With Frictions}
\begin{frame}
\tableofcontents[currentsection] 
\end{frame}

\subsection{A simple Shimer and Smith Model}


\begin{frame}{Matching with Frictions: Environment}
\begin{midi}
\item A fixed measure of workers indexed by $x \in \mathbb{X}$ (uniform)
\item A fixed measure of jobs indexed by $y \in \mathbb{Y}$ (uniform)
\item A production function $f(x,y)$
\item Common ranking $f_x>0, f_y>0$
\item We allow matched agents to transfer each other $w$ (the wage).
\item unemployed get $b(x)$; vacancies cost $c(y)$
\item workers and firms care about EPV (forward looking)
\end{midi}
\end{frame}

\begin{frame}{Matching with Frictions: Allocations}
\begin{midi}
\item $u(x)$ is the mass of unemployed workers, $v(x)$ is the mass of vacancies
\item $h(x,y)$ is the mass of matches (like $\mu$, but not pure anymore!)
\item $w(x,y)$ is the wage and $M(x,y)$ the matching decision (yes/no)
\end{midi}
\end{frame}

\begin{frame}{Matching Process}
\begin{widei}
\item Meeting technology is imperfect:
\begin{itemize}
\item unemployed find offers at rate $\lambda$
\item vacancies find workers at rate $\mu$
\item $\lambda$ and $\mu$ can be endogenized with a matching function:
\begin{itemize}
	\item the number of matches is $N=m(U,V)$
	\item then $\lambda = \frac{N}{U}, \mu=\frac{N}{V}$
	\item a classic matching function is $m(u,v)=\alpha u^{0.5} v^{0.5}$
\end{itemize}
\end{itemize}
\item matching is random: workers draw from $v(y)$, firms draw from $u(x)$
\end{widei}
\end{frame}

\begin{frame}{Matching Process: Timing}
\begin{widee}
\item production: matches produce output and pay wage
\item meeting: U and V meet
\item matching: newly matched pairs decide wether to start partnership
\item separation: existing matches destroyed at rate $\delta$
\end{widee}
\end{frame}



\begin{frame}{Match Surplus - Present Values}

\begin{midi}
\item $W_1(x,y,w)$ and $W_0(x)$ are EPV of employed and unemployed
\item $\Pi_1(x,y,w)$ and $\Pi_0(y)$ are EPV of job and vacancy
\item Surplus is defined as
\begin{equation*}
S(x,y) := W_1(x,y,w) + \Pi_1(x,y,w) - W_0(x) - \Pi_0(y)
\end{equation*}
\item Worker EPV: $rW_1(x,y,w) = w + \delta(W_0(x) -W_1(x,y,w)) $
\item Job EPV:    $r\Pi_1(x,y,w) = f(x,y) -w  + \delta(\Pi_0(y) -\Pi_1(x,y,w)) $
\end{midi}
\end{frame}

\begin{frame}{Value of Match Surplus}
Some simple algebra gives us that:
\begin{equation*}
(r+\delta) S(x,y) = f(x,y) - rW_0(x) - r\Pi_0(y)
\end{equation*}
\begin{midi}
\item Note that we don't need to know the wage to compute this!
\item Under TU, the matching decision is $M(x,y) = \mathbf{1}[S(x,y)\geq0]$
\item Surplus can be non-monotonic because of option value!
\item Surplus inherits complementarity directly from $f$.
\end{midi}
\end{frame}

\begin{frame}{Wages and Division of Surplus}
\begin{midi}
\item There an infinite number of ways to split the surplus
\item S-S assume: nash bargaining with $\alpha$ the worker's bargaining power.
\item then the optimal wage $w(x,y)$ solves
\begin{equation*}
(1-\alpha)\left(W_1(x,y,w) - W_0(y) \right) = \alpha\left(\Pi_1(x,y,w) - \Pi_0(y)\right)
\end{equation*}
\pause
\item Therefore, upon meeting
\begin{itemize}
\item worker gets $W_0(x) + \alpha(S(x,y)$
\item firm gets $\Pi_0(x) + (1-\alpha)(S(x,y)$
\end{itemize}
\end{midi}
\end{frame}

\begin{frame}{EPV of unemployed and vacancy}
\begin{widei}
\item EPV of the unemployed:
\begin{equation*}
rW_0(x) = (1+r)b(x) + \lambda \int \alpha M(x,y) S(x,y) \frac{v(y)}{V}dy
\end{equation*}
\item EPV of a vacancy:
\begin{equation*}
r\Pi_0(x) = -(1+r)c(y) + \mu \int (1-\alpha) M(x,y) S(x,y) \frac{u(x)}{U}dx
\end{equation*}
\item Matching Distribution
\begin{equation*}
\delta h(c,y) = \frac{\lambda}{V}M(x,y)u(x)v(y)
\end{equation*}
\end{widei}
\end{frame}

\begin{frame}{Equlibrium}
Given the primitives $f(x,y),c(y),b(x),r,\delta,\alpha, \lambda,\mu$, a stationary search equilibrium is defined by
\begin{midi}
\item EPVs: $S(x,y),\Pi_0,W_0,\Pi_1,\Pi_0$
\item Allocations: $h(x,y),u(x),v(y)$
\item wage $w(x,y)$ and matching functions $M(x,y)$
\end{midi}
such that
\begin{mide}
\item the EPVs solve the Bellman Equations
\item the wage is the Nash barginaing solution
\item the distributions satisfy stationarity and adding up propoerties.
\end{mide}
\end{frame}

\begin{frame}{Results}
\begin{midi}
\item Existence: Yes \cite{10.2307/2999430}
\item Uniqueness: NO
\item Efficiency: Not in general
\begin{itemize}
\item workers do not internalize how the affect others' search (search externality)
\item romm for efficiency improving policies
\end{itemize}
\item Assortative Matching
\begin{itemize}
\item \cite{10.2307/2999430} introduce new definitions: monotonicity of matching set boundaries.
\item log supermodular $f(x,y)\rightarrow $ PAM
\item log submodular $f(x,y)\rightarrow $ NAM
\item this requires stronger complementarities than in frictionless world.
\end{itemize}
\end{midi}

\end{frame}

\begin{frame}[allowframebreaks]{References}
        % \frametitle{References}
		\bibliographystyle{unsrtnat}
		\bibliography{labor2017.bib}
\end{frame}


\end{document}
